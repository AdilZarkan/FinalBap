%%=============================================================================
%% Samenvatting
%%=============================================================================

% TODO: De "abstract" of samenvatting is een kernachtige (~ 1 blz. voor een
% thesis) synthese van het document.
%
% Deze aspecten moeten zeker aan bod komen:
% - Context: waarom is dit werk belangrijk?
% - Nood: waarom moest dit onderzocht worden?
% - Taak: wat heb je precies gedaan?
% - Object: wat staat in dit document geschreven?
% - Resultaat: wat was het resultaat?
% - Conclusie: wat is/zijn de belangrijkste conclusie(s)?
% - Perspectief: blijven er nog vragen open die in de toekomst nog kunnen
%    onderzocht worden? Wat is een mogelijk vervolg voor jouw onderzoek?
%
% LET OP! Een samenvatting is GEEN voorwoord!

%%---------- Nederlandse samenvatting -----------------------------------------
%
% TODO: Als je je bachelorproef in het Engels schrijft, moet je eerst een
% Nederlandse samenvatting invoegen. Haal daarvoor onderstaande code uit
% commentaar.
% Wie zijn bachelorproef in het Nederlands schrijft, kan dit negeren, de inhoud
% wordt niet in het document ingevoegd.

\IfLanguageName{english}{%
\selectlanguage{dutch}
\chapter*{Samenvatting}

\selectlanguage{english}
}{}

%%---------- Samenvatting -----------------------------------------------------
% De samenvatting in de hoofdtaal van het document

\chapter*{\IfLanguageName{dutch}{Samenvatting}{Abstract}}

Het doel van deze studie is om de probleemstelling rond 'Is het MS Bot Framework matuur genoeg om complexe/dynamische business processen binnen MS Dynamics 365FO te automatiseren (a), en is dit nuttig (b).

Deze studie is vooral interessant voor bedrijven die oplossingen in MS Dynamics 365 for Finance and Operations aanbieden. Zo kan er heel specifiek op basis van elke business case bekeken worden voor welke onderdelen dergelijke bot implementatie nuttig zou kunnen zijn. Ook voor bedrijven die oplossingen in het MS Bot Framework aanbieden is dit een relevante studie. Het zal immers in beperkte mate ook onderzoeken hoe matuur en gebruiksklaar de software is voor complexere processen.
 
Als fundering werd gekozen voor Dynamics 365 for Finance and Operations, wat een heel uitgebreide ERP is, en dus veel complexe vraagstukken door middel van dynamische bedrijfsprocessen omvat en oplost. Het chatbot framework dat hier een oplossing voor zou moeten kunnen bieden is het Microsoft Bot Framework.

Uiteraard is het niet mogelijk om heel MS D365FO te gaan vervangen door een chatbot. Dit omdat D365FO een product is dat al meer dan 20 jaar wordt ontwikkeld, en dus zo ruim is dat dit simpelweg niet haalbaar zou zijn. Daarom werd er voor het beoordelen van onderzoeksvraag A gekozen om een POC te ontwikkelen voor Dynamics 365FO. Aan de hand van die POC zal dan een bevraging worden gedaan naar meningen van gebruikers die hiermee in contact komen, om een advies te kunnen formuleren voor onderzoeksvraag B. 

Zo zal een poging worden ondernomen om de Job Card Device module te proberen automatiseren/optimaliseren door een chatbotapplicatie (POC).De module (waarvan u de beschrijving kunt lezen in sectie 5.1) zou dus vervangbaar moeten zijn door het POC, die eveneens communiceert met D365FO. Dit kan men bekijken als een verandering voor de gebruiker, maar achter de schermen zou alles hetzelfde moeten gebeuren (de zelfde data stroomt door D365FO, maar op een andere manier). 

Het POC zal ontwikkeld worden in het MS Bot Framework. Alvorens dit kan gebeuren, moet er een requirementsanalyse opgemaakt worden. Hierin staan alle functionaliteiten die de bot moet kunnen overnemen. Vervolgens zal aan de hand van het POC een antwoord geformuleerd worden op onderzoeksvraag A. . 

Het tweede deel van dit onderzoek zal proberen in kaart brengen of dit nuttig is. Dit zal in eerste instantie gebeuren door een technische evaluatie waarin de voor- en nadelen van het framework onderzocht zullen worden. In tweede instantie zal er ook een bevraging worden opgenomen waarin getoetst zal worden of gebruikers tevreden zijn met dergelijke verandering, en of ze hier een nut/toekomst in zien.
 
Ten slotte zal er ook een bevraging gedaan worden bij 20 subjects. Aan de hun van hun meningen en opmerkingen zal ook voor onderzoeksvraag B een antwoord en advies geformuleerd worden. 

Na het lezen van dit onderzoek zou men dus inzicht verworven moeten hebben in de integratiemogelijkheden van het MS Bot Framework binnen Dynamics 365FO (a) en het nut hiervan (b). 

Aan het einde van deze bachelorproef zal geconcludeerd worden dat het Microsoft Bot Framework wel degelijk matuur genoeg is om dergelijke bedrijfsprocessen over te nemen. Echter zal wel ook een advies geformuleerd worden waarin wordt aanbevelen om de bot eerder te gaan implementeren voor data weergave dan voor data input. 









