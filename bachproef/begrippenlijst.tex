%%=============================================================================
%% Begrippenlijst
%%=============================================================================
\chapter{Begrippenlijst}
\label{ch:begrippenlijst}

\section{Gebruikte afkortingen}
\begin{tabular}{|c|c|}
    \hline 
    D365FO &  Microsoft Dynamics 365 for Finance and Operations \\
    \hline 
    MBS & Microsoft Business Solutions \\
    \hline
    CRM & Customer Relationship Management  \\
    \hline
    CEO & Chief Executive Officer \\
    \hline
    AI & Artificiële Intelligentie \\
    \hline
    NLP & Natural Language Processing \\
    \hline
    ERP & Enterprise Resource Planning \\
    \hline
    POC & Proof of concept \\
    \hline 
    BOM & Bill of Materials \\
    \hline 
    BPMN & Business Process Model and Notation \\
    \hline 
    (Azure) AAD & Azure Active Directories \\
    \hline 
    SDK & Software Development Kit \\
    \hline
    FaaS & Function as a Service \\
    \hline
\end{tabular} 

\subsection{Begrippenlijst}
\subsubsection{User stories}
Korte, simpele beschrijvingen van nieuwe functionele requirements vanuit het perspectief van de gebruiker. (\textcite{Cohn2004})

\subsubsection{Enterprise Resource Planning}
Geïntegreerde informatiesystemen, gebouwd op een gecentraliseerde database die op eenzelfde computer platform draait als de diverse resources van het bedrijf, en op die manier de flow van informatie (over deze resources) tussen alle business functies van het bedrijf faciliteert. (\textcite{McGraw-Hill2011})

\subsubsection{Stateful <> Stateless}
Een stateless server heeft, in tegenstelling tot een stateful server, geen notie van een sessie. Dit wil zeggen: de volledige periode dat een gebruiker een site raadpleegt, van start bezoek tot einde bezoek (sluiten van het tabblad). Een site waar men bijvoorbeeld kan inloggen gebruikt een sessie. Gedurende deze sessie moet de site in kwestie immers onthouden dat de gebruiker ingelogd is, en zijn credentials onthouden zodat hij niet telkens opnieuw moet inloggen bij elke klik. Met andere woorden: een stateful server onthoudt informatie tussen requests in. Een stateless server doet dit niet. (\textcite{Subhash2009}) 

\subsubsection{Serverless computing (FaaS))}
Serverless computing, ook wel gekend als Function as a Service (FaaS), is een technologie die de cloud provider compleet beheer verschaft. Wanneer developers hier gebruik van maken, hoeven ze geen rekening te houden met upscaling etc. (\textcite{Chowan2018})






