%%=============================================================================
%% Begrippenlijst
%%=============================================================================
\chapter{Begrippenlijst}
\label{ch:begrippenlijst}

\section{Gebruikte afkortingen}
\begin{tabular}{|c|c|}
    \hline 
    D365FO &  Microsoft Dynamics 365 for Finance and Operations \\
    \hline 
    MBS & Microsoft Business Solutions \\
    \hline
    CRM & Customer Relationship Management  \\
    \hline
    CEO & Chief Executive Officer \\
    \hline
    AI & Artificiële Intelligentie \\
    \hline
    NLP & Natural Language Processing \\
    \hline
    ERP & Enterprise Resource Planning \\
    \hline
    POC & Proof of concept \\
    \hline 
\end{tabular} 

\subsection{Begrippenlijst}
\subsubsection{User stories}
Korte, simpele beschrijvingen van nieuwe functionele requirements vanuit het perspectief van de gebruiker. (\textcite{Cohn2004})

\subsubsection{Enterprise Resource Planning}
Geïntegreerde informatiesystemen, gebouwd op een gecentraliseerde database die op eenzelfde computer platform draait als de diverse resources van het bedrijf, en op die manier de flow van informatie (over deze resources) tussen alle business functies van het bedrijf faciliteert. (\textcite{Hill2011})

\subsubsection{Stateless, states}

\subsubsection{Serverless}




