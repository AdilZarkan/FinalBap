%%=============================================================================
%% Stakeholders
%%=============================================================================

\chapter{Stakeholders}
\label{ch:conclusie}

Aangezien dit onderzoek werd uitgevoerd in opdracht van Delaware, is het niet onbelangrijk om te schetsen wat dit bedrijf juist is. 

\section{Delaware}
\subsection{Algemeen}
Delaware is een jonge consultancy firma die geavanceerde oplossingen op maat maakt voor, en diensten aanbiedt aan bedrijven die hun competitieve positie willen versterken op een houdbare manier. Zo zijn ze gespecialiseerd in het transformeren van hun klanten hun Finance, HR, Communicatie, en IT afdelingen. Dit varieert tussen implementaties op maat in vaak nieuwe technologieën, tot meedenken op strategische niveaus om hun brand te verbeteren. 
Sinds de privatisering in 2003 zijn hun business activiteiten fors gegroeid. Zo telt Delaware op de dag van vandaag meer dan 2000 werknemers, van 32 nationaliteiten, verspreid over 24 kantoren in 12 landen. In 2017 onderging het een naamsverandering, zo werd Delaware Consulting uiteindelijk Delaware. Dit toont hun engagement om aan te passen aan veranderende klantnoden en omstandigheden, die vaak de capaciteiten van een reguliere IT-consultancy firma overstijgen. 
Delaware steunt al sinds het prille begin op 3 pilaren. Allereerst Operational Excellence, hiermee bedoelt men dat ze op operationeel vlak, in de mate van het mogelijke, steeds zo efficiënt mogelijk te werk willen gaan, wat zich vaak uit in het verkennen van nieuwe technologieën en implementaties op maat. Ten tweede Business Insights, want technologie moet nu eenmaal als meerwaarde hebben dat de business waardevolle inzichten kan verwerven dankzij die technologie om met die inzichten aanpassingen te kunnen maken. Tenslotte is ook Customer Experience een pilaar, daar klanten de sleutel zijn voor elke firma. Deze 3 pilaren zijn sterk te herkennen wanneer men het heeft over de specialisaties van Delaware, namelijk ERP-systemen, Business Intelligence en E-commerce. 
Als een van de leading partners van SAP en Microsoft beschikken ze over meerdere certificaten, zoals: Gold Certified Microsoft, Gold SAP, Gold Business Objects and Platinum OpenText partners.

\subsection{Vestigingen}
De dag van vandaag telt Delaware 5 vestigingen in België, met als hoofdzetel de vestiging in Kortrijk. Dit omdat het bedrijf bewust omspringt met de behoeften van klanten, maar vooral van werknemers. Het is niet zinvol om je werknemers elke dag de file in te sturen, dus proberen ze als firma dichter bij hun werknemers aanwezig te zijn. 
Ook op globaal vlak deinst het bedrijf hier niet voor terug. Zo zijn veel van de internationale vestigingen strategisch gekozen om mee te groeien met hun klanten, en vooral hun internationale noden. 

\subsection{Bedrijfshiërarchie }
Delaware gaat vol voor gLocal partnerships. Dit is een combinatie van een globale visie, die zich uit in de meerdere vestigingen over verschillende continenten, zonder lokaal ondernemerschap uit het oog te verliezen. Zo worden alle medewerkers gestimuleerd om initiatief te nemen en worden voorstellen steeds met grote belangstelling 

\section{Microsoft}
%te bespreken => 2e leven microsoft (erp, business2business, open-source,...)
