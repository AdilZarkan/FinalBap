%%=============================================================================
%% Stakeholders
%%=============================================================================

\chapter{Stakeholders}
\label{ch:conclusie}


\section{delaware}
\subsection{Algemeen}
delaware is een jonge consultancy firma die geavanceerde oplossingen op maat maakt voor, en diensten aanbiedt aan bedrijven die hun competitieve positie willen versterken op een houdbare manier. Zo zijn ze gespecialiseerd in het transformeren van hun klanten hun Finance, HR, Communicatie, en IT afdelingen. Dit varieert tussen implementaties op maat in vaak nieuwe technologieën, tot meedenken op strategische niveaus om hun brand te verbeteren. 
Sinds de herstructurering in 2005 zijn hun business activiteiten fors gegroeid. Zo telt delaware op de dag van vandaag meer dan 2000 werknemers, van 32 nationaliteiten, verspreid over 24 kantoren in 12 landen. In 2017 onderging het een naamsverandering, zo werd delaware Consulting uiteindelijk delaware. Dit toont hun engagement om aan te passen aan veranderende klantnoden en omstandigheden, die vaak de capaciteiten van een reguliere IT-consultancy firma overstijgen. Zo denkt delaware ook mee op de strategische en operationele niveaus.  
delaware steunt al sinds het prille begin op 3 pilaren. Allereerst Operational Excellence, hiermee bedoelt men dat ze op operationeel vlak, in de mate van het mogelijke, steeds zo efficiënt mogelijk te werk willen gaan, wat zich vaak uit in het verkennen van nieuwe technologieën en implementaties op maat. Ten tweede Business Insights, want technologie moet nu eenmaal als meerwaarde hebben dat de business waardevolle inzichten kan verwerven dankzij die technologie om met die inzichten aanpassingen te kunnen maken. Tenslotte is ook Customer Experience een pilaar, daar klanten de sleutel zijn voor elke firma. Deze 3 pilaren zijn sterk te herkennen wanneer men het heeft over de specialisaties van delaware, namelijk ERP-systemen, Business Intelligence en E-commerce. 
Als een van de leading partners van SAP en Microsoft beschikken ze over meerdere certificaten, zoals: Gold Certified Microsoft, Gold SAP, Gold Business Objects and Platinum OpenText partners.

\subsection{Vestigingen}
De dag van vandaag telt delaware 5 vestigingen in België, met als hoofdzetel de vestiging in Kortrijk. Dit omdat het bedrijf bewust omspringt met de behoeften van klanten, maar vooral van werknemers. 
Ook op globaal vlak deinst het bedrijf hier niet voor terug. Zo zijn veel van de internationale vestigingen strategisch gekozen om mee te groeien met hun klanten, en vooral hun internationale noden. 

\section{Microsoft}


\subsection{MS Bot Framework} 
Als de integratie tussen het Bot Framework en Dynamics 365 FO van dermate hoge kwaliteit is zal ook Microsoft hier voordeel uit putten. Zo zal hun ERP systeem een aanzienlijke toegevoegde waarde krijgen, dankzij hun eigen chatbot framework. Het is daarom niet onbelangrijk wat de uitslag en conclusie zal zijn van dit onderzoek voor Microsoft. 

\subsection{MS Azure} 
Aangezien het Bot Framework heel nauw verbonden is met Azure (zie volgende hoofdstukken), is ook het hosting platform in enige vorm stakeholder van dit onderzoek.
%te bespreken => 2e leven microsoft (erp, business2business, open-source,...)
