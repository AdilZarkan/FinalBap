%%=============================================================================
%% Voorwoord
%%=============================================================================

\chapter*{\IfLanguageName{dutch}{Woord vooraf}{Preface}}
\label{ch:voorwoord}

%% TODO:
%% Het voorwoord is het enige deel van de bachelorproef waar je vanuit je
%% eigen standpunt (``ik-vorm'') mag schrijven. Je kan hier bv. motiveren
%% waarom jij het onderwerp wil bespreken.
%% Vergeet ook niet te bedanken wie je geholpen/gesteund/... heeft

Met deze bachelorproef wens ik mijn opleiding toegepaste informatica aan de Hogeschool van Gent af te sluiten. De opleiding was zeer leerrijk, en heeft me erg goed klaargestoomd voor de volgende stap in mijn professionele carrière als IT'er, waarvoor dank. Uiteraard wordt een werk van deze omvang nooit alleen geschreven, en wil ik graag van deze kans gebruik maken om enkele mensen te bedanken die me dit hebben helpen realiseren. 

Allereerst wens ik mijn promotor Koen Mertens te bedanken voor de begeleiding, en zijn snelle antwoorden wanneer ik vragen had. Dankzij een eerder moeilijk meetbare onderzoeksvraag verliep vooral de opzet van dit onderzoek stroef. Echter kon meneer Mertens op tijd bijsturen, waarvoor dank.

 Verder ben ik ook het bedrijf delaware, met in het bijzonder mijn co-promotor Andy Marijs en begeleider Ward Lanssens erg dankbaar voor de opportuniteit. Vooral Ward, die me bij zeer veel technische vraagstukken heeft bijgestaan was cruciaal voor het goede verloop van dit onderzoek. 

Mijn voorkennis van zowel het MS Bot Framework als MS Dynamics 365FO was nihil bij het opstarten van deze scriptie, maar dankzij de stevige begeleiding doorheen het hele proces heb ik het tot een goed einde kunnen brengen. Dit deels dankzij leergierigheid van mijn zijde, maar ook dankzij de stevige expertise die men bij delaware door de jaren heen heeft opgebouwd. 

Dankzij deze scriptie heb ik een inzicht verkregen in het Bot Framework enerzijds, maar vooral in Microsoft Dynamics 365FO. Aangezien ik in deze branche ga werken, is de kennis die ik doorheen dit project heb opgebouwd erg nuttig voor het verdere verloop van mijn professionele carrière als business analist binnen het Microsoft Dynamics 365FO Development team binnen delaware. 

