%==============================================================================
% Sjabloon onderzoeksvoorstel bachelorproef
%==============================================================================
% Gebaseerd op LaTeX-sjabloon ‘Stylish Article’ (zie voorstel.cls)
% Auteur: Jens Buysse, Bert Van Vreckem
%
% Compileren in TeXstudio:
%
% - Zorg dat Biber de bibliografie compileert (en niet Biblatex)
%   Options > Configure > Build > Default Bibliography Tool: "txs:///biber"
% - F5 om te compileren en het resultaat te bekijken.
% - Als de bibliografie niet zichtbaar is, probeer dan F5 - F8 - F5
%   Met F8 compileer je de bibliografie apart.
%
% Als je JabRef gebruikt voor het bijhouden van de bibliografie, zorg dan
% dat je in ``biblatex''-modus opslaat: File > Switch to BibLaTeX mode.

\documentclass{voorstel}

\usepackage{lipsum}

%------------------------------------------------------------------------------
% Metadata over het voorstel
%------------------------------------------------------------------------------

%---------- Titel & auteur ----------------------------------------------------

% TODO: geef werktitel van je eigen voorstel op
\PaperTitle{Titel voorstel}
\PaperType{Onderzoeksvoorstel Bachelorproef 2018-2019} % Type document

% TODO: vul je eigen naam in als auteur, geef ook je emailadres mee!
\Authors{Steven Stevens\textsuperscript{1}} % Authors
\CoPromotor{Piet Pieters\textsuperscript{2} (Bedrijfsnaam)}
\affiliation{\textbf{Contact:}
  \textsuperscript{1} \href{mailto:steven.stevens.u1234@student.hogent.be}{steven.stevens.u1234@student.hogent.be};
  \textsuperscript{2} \href{mailto:piet.pieters@acme.be}{piet.pieters@acme.be};
}

%---------- Abstract ----------------------------------------------------------

\Abstract{Hier schrijf je de samenvatting van je voorstel, als een doorlopende tekst van één paragraaf. Wat hier zeker in moet vermeld worden: \textbf{Context} (Waarom is dit werk belangrijk?); \textbf{Nood} (Waarom moet dit onderzocht worden?); \textbf{Taak} (Wat ga je (ongeveer) doen?); \textbf{Object} (Wat staat in dit document geschreven?); \textbf{Resultaat} (Wat verwacht je van je onderzoek?); \textbf{Conclusie} (Wat verwacht je van van de conclusies?); \textbf{Perspectief} (Wat zegt de toekomst voor dit werk?).

Bij de sleutelwoorden geef je het onderzoeksdomein, samen met andere sleutelwoorden die je werk beschrijven.

Vergeet ook niet je co-promotor op te geven.
}

%---------- Onderzoeksdomein en sleutelwoorden --------------------------------
% TODO: Sleutelwoorden:
%
% Het eerste sleutelwoord beschrijft het onderzoeksdomein. Je kan kiezen uit
% deze lijst:
%
% - Mobiele applicatieontwikkeling
% - Webapplicatieontwikkeling
% - Applicatieontwikkeling (andere)
% - Systeembeheer
% - Netwerkbeheer
% - Mainframe
% - E-business
% - Databanken en big data
% - Machineleertechnieken en kunstmatige intelligentie
% - Andere (specifieer)
%
% De andere sleutelwoorden zijn vrij te kiezen

\Keywords{Onderzoeksdomein. Keyword1 --- Keyword2 --- Keyword3} % Keywords
\newcommand{\keywordname}{Sleutelwoorden} % Defines the keywords heading name

%---------- Titel, inhoud -----------------------------------------------------

\begin{document}

\flushbottom % Makes all text pages the same height
\maketitle % Print the title and abstract box
\tableofcontents % Print the contents section
\thispagestyle{empty} % Removes page numbering from the first page

%------------------------------------------------------------------------------
% Hoofdtekst
%------------------------------------------------------------------------------

% De hoofdtekst van het voorstel zit in een apart bestand, zodat het makkelijk
% kan opgenomen worden in de bijlagen van de bachelorproef zelf.
%---------- Inleiding ---------------------------------------------------------

\section{Introductie} % The \section*{} command stops section numbering
\label{sec:introductie}
Delaware is een vooruitstrevende organisatie. Dit wil zeggen dat men steeds bezig is met het onderzoeken en ontwikkelen van "next-functionality"  zodat ze klaar zijn om in productie te gaan van zodra de markten hier rijp voor zijn. Echter blijft de Data-inputting voor vele organisaties een lastig onderdeel dat veel tijd en middelen in beslag neemt. Het gebruiken van Cognitive services en bots klinkt hiervoor zeer veelbelovend, maar zal het gebruik hiervan ook binnen een professionele ERP-omgeving nog kunnen zorgen voor toegevoegde waarde? Dit zal onderzocht worden enerzijds, maar anderzijds zal er ook een proof of concept gemaakt worden binnen Microsoft Dynamics ERP. 


%---------- Stand van zaken ---------------------------------------------------

\section{State-of-the-art}
\label{sec:state-of-the-art}
Er zijn reeds gelijkaardige onderzoeken gevoerd naar de integratie van cognitive services in ERP. Zo is er~\autocite{RizkyDharana2017}, de conclusie hiervan was dat de integratie tussen chatbots en ERP-systemen managers zullen helpen om effectiever en efficiënter te voldoen aan klantbehoeften. Echter werd hier geen onderzoek naar de rendabiliteit van genoemde integraties. In een ander werk ~\autocite{Baat2016} concludeerde men de mix tussen mensen - processen - technologie drastisch zal veranderen omwille van AI. Zo zullen veel (data-inputting) taken die tot op vandaag door mensen worden verricht, geautomatiseerd kunnen worden. Omdat een winstgevende organisatie, zoals Delaware, steeds als doelstelling heeft om operationele kosten te minimaliseren, is het daarom belangrijk dat dit onderzoek gevoerd wordt naar de rendabiliteit van de integratie tussen ERP en Cognitive services \& AI. 
% Voor literatuurverwijzingen zijn er twee belangrijke commando's:
% \autocite{KEY} => (Auteur, jaartal) Gebruik dit als de naam van de auteur
%   geen onderdeel is van de zin.
% \textcite{KEY} => Auteur (jaartal)  Gebruik dit als de auteursnaam wel een
%   functie heeft in de zin (bv. ``Uit onderzoek door Doll & Hill (1954) bleek
%   ...'')
%---------- Methodologie ------------------------------------------------------
\section{Methodologie}
\label{sec:methodologie}

Dit onderzoek zal verricht worden door de productie van een proof of concept waarin verschillende cognitive services gekoppeld aan AI getest zullen worden in een integratie met Microsoft Dynamics ERP. Dit zal in eerste instantie een chatbot zijn, en kan later uitgebreid worden. 

%---------- Verwachte resultaten ----------------------------------------------
\section{Verwachte resultaten}
\label{sec:verwachte_resultaten}

Ik verwacht dat de integratie met ERP zal lukken, aangezien hier al voorbeelden van zijn.  Daarentegen verwacht ik niet dat de technologie al klaar is om heel complexe en dynamische data-inputting taken te automatiseren. Echter moet het volgens mij wel mogelijk zijn om (relatief) simpele intents te herkennen en verwerken.  

%---------- Verwachte conclusies ----------------------------------------------
\section{Verwachte conclusies}
\label{sec:verwachte_conclusies}

De onderzoeksvraag is in dit geval zeer tijdsgebonden. Het is geen vraag meer of cognitive services, gekoppeld aan AI, een meerwaarde zullen bieden aan B2B softwaresystemen, met name ERP. Echter ben ik er niet van overtuigd dat de technologie hier al klaar voor is. Daarom denk ik dat er nog te veel bedrijfseigen aanpassingen zullen moeten gebeuren per business case waarop deze services worden toegepast, en daarom moeilijk rendabel zullen zijn. 



%------------------------------------------------------------------------------
% Referentielijst
%------------------------------------------------------------------------------
% TODO: de gerefereerde werken moeten in BibTeX-bestand ``voorstel.bib''
% voorkomen. Gebruik JabRef om je bibliografie bij te houden en vergeet niet
% om compatibiliteit met Biber/BibLaTeX aan te zetten (File > Switch to
% BibLaTeX mode)

\phantomsection
\printbibliography[heading=bibintoc]

\end{document}
